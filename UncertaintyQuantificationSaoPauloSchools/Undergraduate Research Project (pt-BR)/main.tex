\documentclass{article}
%encoding
%--------------------------------------
\usepackage[T1]{fontenc}
\usepackage[utf8]{inputenc}
%--------------------------------------
 
%Portuguese-specific commands
%--------------------------------------
\usepackage[portuguese]{babel}
%--------------------------------------
 
%Hyphenation rules
%--------------------------------------
\usepackage{hyphenat}
\hyphenation{mate-mática recu-perar}
%--------------------------------------


\title{Avaliação da Qualidade de Serviços de Geocodificação Online em Cidades Brasileiras}

\author{Orientador: Prof. Rodrigo Cesar Pedrosa Silva \\
Bolsista: Ana Luiza Almeida Soares}

\date{\today}

\usepackage{natbib}
\usepackage{graphicx}

\begin{document}

\maketitle

\section{Resumo}
Este projeto de pesquisa concentra-se em estimar a qualidade dos serviços de geocodificação online adaptados para cidades brasileiras. A geocodificação, o processo de transformar endereços em coordenadas geográficas, é crucial para uma ampla gama de aplicações em transporte, planejamento urbano, serviços de emergência e operações comerciais. No entanto, a qualidade dos resultados de geocodificação varia entre diferentes provedores de serviço e regiões, tornando necessário avaliar e comparar esses serviços.

O projeto tem como objetivo avaliar a precisão, de diferentes serviços de geocodificação online na correta geocodificação de endereços em cidades brasileiras. Será coletado um conjunto diversificado de dados que incluirá endereços urbanos e rurais com diferentes complexidades. Coordenadas de referência obtidas de fontes confiáveis serão usadas para validar a precisão dos serviços de geocodificação. A pesquisa estabelecerá métricas quantitativas para avaliar e comparar o desempenho dos serviços de geocodificação.

Os resultados deste projeto de pesquisa terão implicações significativas para análise espacial, processos de tomada de decisão, serviços baseados em localização, desenvolvimento de infraestrutura e operações comerciais em cidades brasileiras. Os resultados permitirão que pesquisadores, formuladores de políticas e profissionais tomem decisões informadas sobre a seleção e utilização dos serviços de geocodificação. Além disso, o projeto preenche a lacuna de pesquisa específica para o contexto brasileiro e contribui para o conhecimento existente em pesquisa de geocodificação. Em última análise, este projeto de pesquisa visa aprimorar a precisão, eficiência e confiabilidade dos serviços de geocodificação para cidades brasileiras, beneficiando diversos setores e partes interessadas envolvidas em aplicações de geocodificação.
   
\section{Palavras-chave}
Geocodificação, Análise espacial, Serviços de geocodificação online
   
\section{Introdução}
	
Geocodificação, o processo de transformar endereços em coordenadas geográficas, é um componente fundamental dos serviços de localização e da análise espacial moderna. Com a crescente demanda por informações de geolocalização precisas e eficientes, os serviços de geocodificação online se tornaram ferramentas cruciais para uma ampla variedade de aplicações, incluindo sistemas de navegação, plataformas de mídia social, gestão logística, planejamento urbano e muitas outras. No entanto, a qualidade e confiabilidade dos resultados de geocodificação dos diferentes provedores de serviço pode variar devido a diferenças nos algoritmos subjacentes, fontes de dados e técnicas de processamento, tornando essencial avaliar e comparar o desempenho desses serviços.

Este projeto de pesquisa tem como objetivo estimar a qualidade de diferentes serviços de geocodificação online para cidades brasileiras. O Brasil, sendo um país vasto e diverso com um sistema de endereçamento complexo, apresenta desafios únicos para as aplicações de geocodificação. A precisão e a exatidão dos serviços de geocodificação nesse contexto têm implicações significativas para vários setores, como transporte, serviços de emergência e operações comerciais. Portanto, compreender os pontos fortes e limitações dos serviços de geocodificação disponíveis é de extrema importância tanto para a pesquisa acadêmica quanto para as aplicações práticas.	
   
\section{Objetivos}

O objetivo principal deste projeto de pesquisa é desenvolver uma metodologia abrangente para estimar a qualidade de diferentes serviços de geocodificação online para cidades brasileiras. Ao avaliar e comparar rigorosamente seu desempenho, pretendemos fornecer insights valiosos sobre os pontos fortes e limitações de cada serviço, permitindo que os usuários façam escolhas informadas com base em seus requisitos específicos. O projeto abrangerá uma variedade de dimensões, incluindo precisão, completude, velocidade, confiabilidade e facilidade de integração, para fornecer uma avaliação holística da qualidade dos serviços de geocodificação.
   
\section{Justificativa/Relevância}

As justificativas para este projeto podem ser colocadas nos seguintes termos:

1. Atendendo a uma Necessidade Crítica:

Serviços de geocodificação precisos e confiáveis são essenciais para uma ampla gama de aplicações em diversos setores, incluindo transporte, planejamento urbano, serviços de emergência e operações comerciais. No contexto das cidades brasileiras, que possuem um sistema de endereçamento complexo e características geográficas diversas, a necessidade de soluções robustas de geocodificação torna-se ainda mais crítica. Este projeto de pesquisa aborda essa necessidade avaliando e comparando a qualidade de diferentes serviços de geocodificação aplicados à cidades brasileiras.

2. Aprimorando a Análise Espacial e a Tomada de Decisão:

Erros de geocodificação podem ter consequências significativas nos processos de análise espacial e tomada de decisão. Resultados de geocodificação imprecisos ou inexatos podem levar a análises falhas, decisões equivocadas e alocação ineficiente de recursos. Ao estimar a qualidade dos serviços de geocodificação online, este projeto de pesquisa fornece insights valiosos a pesquisadores, formuladores de políticas e profissionais, permitindo que tomem escolhas mais informadas ao utilizar serviços de geocodificação para análises espaciais, planejamento urbano e processos de tomada de decisão.

3. Apoio ao Desenvolvimento de Infraestrutura:

A geocodificação precisa é crucial para iniciativas de desenvolvimento de infraestrutura, incluindo redes de transporte, serviços de utilidade pública e sistemas de resposta a emergências. Ao avaliar a precisão dos serviços de geocodificação, este projeto de pesquisa fornece insights valiosos para planejadores e formuladores de políticas de infraestrutura. Os resultados podem contribuir para o desenvolvimento de sistemas de endereçamento robustos, algoritmos de roteamento eficientes e aprimoramento das capacidades de resposta a emergências, aprimorando assim o desenvolvimento geral da infraestrutura nas cidades brasileiras.

4. Preenchendo a Lacuna na Pesquisa:

Embora a pesquisa em geocodificação tenha sido amplamente estudada em vários contextos, há uma falta de avaliações abrangentes específicas para as cidades brasileiras \cite{}. Este projeto de pesquisa preenche a lacuna na pesquisa ao se concentrar nos desafios únicos e nos requisitos da geocodificação no contexto brasileiro. Os resultados contribuirão para o conhecimento existente na pesquisa de geocodificação, enriquecendo a compreensão da qualidade e desempenho da geocodificação em sistemas geográficos e de endereçamento diversos.
   
\section{Atividades/Metodologias}

Para alcançar nossos objetivos de pesquisa, seguiremos uma metodologia sistemática e abrangente. Os seguintes passos serão realizados:

1. Coleta de Dados: Vamos reunir um conjunto de dados diversificado e representativo, consistindo em endereços de várias cidades brasileiras. Este conjunto de dados incluirá uma combinação de endereços urbanos e rurais, diferentes formatos de endereço e uma variedade de complexidade de endereços para garantir uma avaliação abrangente.

2. Seleção dos Serviços de Geocodificação: Vamos identificar e selecionar um conjunto de serviços populares de geocodificação online disponíveis para cidades brasileiras. Esses serviços serão escolhidos com base em sua ampla utilização, disponibilidade de APIs documentadas e reputação na comunidade geoespacial.

3. Experimentos de Geocodificação: Utilizando o conjunto de dados coletados, iremos realizar experimentos de geocodificação, enviando os endereços para cada serviço de geocodificação selecionado. Registraremos os resultados de geocodificação, incluindo as coordenadas de latitude e longitude, retornados por cada serviço para análises posteriores.

4. Dados de Referência: Para validar a precisão dos serviços de geocodificação, obteremos coordenadas de referência para um subconjunto de endereços a partir de fontes confiáveis, como bancos de dados governamentais, mapas oficiais ou conjuntos de dados geoespaciais autoritativos. Essas coordenadas de referência servirão como base para avaliar a precisão de cada serviço de geocodificação.

5. Análise e Comparação: Com base nos resultados de geocodificação coletados e nas métricas de avaliação, iremos analisar e comparar o desempenho de diferentes serviços de geocodificação. Essa análise envolverá a identificação de pontos fortes e fracos, determinação dos serviços mais precisos, e compreensão de quaisquer variações regionais no desempenho.

6. Análise Estatística: Empregaremos técnicas estatísticas para validar a significância das diferenças observadas nas métricas de desempenho entre os serviços de geocodificação. Essa análise fornecerá insights sobre a confiabilidade e consistência dos serviços avaliados.

7. Análise de Sensibilidade: Realizaremos análises de sensibilidade para investigar o impacto de fatores variados, como complexidade do endereço, tamanho do conjunto de dados e distribuição geográfica, no desempenho dos serviços de geocodificação. Essa análise ajudará a identificar cenários em que determinados serviços se destacam ou enfrentam dificuldades, proporcionando uma compreensão mais profunda de seus pontos fortes e limitações.

8. Validação e Testes de Robustez: Para garantir a robustez de nossas descobertas, realizaremos testes de validação, como validação cruzada e bootstrap, para avaliar a estabilidade e generalizabilidade dos resultados obtidos.

9. Recomendações: Com base na análise e comparação abrangente dos serviços de geocodificação, forneceremos recomendações sobre os serviços mais adequados para diferentes aplicações e destacaremos áreas para melhoria no cenário de geocodificação para cidades brasileiras.

%\section{Resultados Preliminares}
%(Apenas em caso de renovação)
%(Limite 1500 caracteres)	
   
 

\bibliographystyle{plain}
\bibliography{references}
\end{document}
