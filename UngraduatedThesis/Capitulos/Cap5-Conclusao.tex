\chapter{Considerações Finais} \label{consideracoes}

O presente trabalho apresentou uma análise da qualidade das APIs Mapbox, TomTom e Here para os dados disponibilizados pelo CEM - Centro de estudos da Metrópole. Devido a problemas no Crawler, que é a aplicação que solicita e coleta a geocodificação, tivemos poucas respostas e estas foram insatisfatórias. A conclusão atual é de que as API cometem muitos erros graves e que não há relação clara entre a discrepância e o erro. 

No entanto, quaisquer conclusões tiradas a partir desse estudo são enviesadas a partir do momento em que não temos dados o suficiente e estes são dados específicos. É importante ressaltar que a base de dados possui apenas endereços de escolas, não tendo uma diversidade de imóveis, localizados na região metropolitana de São de Paulo, o que limita a diversidade de localidades consideradas.

Sendo assim, é necessária a repetição do experimento com um maior montante de dados. Para a próxima etapa do trabalho, iremos repetir os experimentos apresentados com uma nova solicitação de geocodificação, além de incluir as APIs faltantes, Google Maps e Open Route Service. Acreditamos que, ao repetir o experimento, possamos compreender melhor o comportamento do erro e comparar os resultados com APIs já consolidadas na academia, como o Google Maps. Além disso, planejamos realizar toda a análise para uma amostra significativa da base de dados da \cite{Prodabel}, que conta com 85 mil endereços distribuídos no espaço. Esperamos que com a maior quantidade de endereços, possamos analisar o comportamento de forma mais clara. Em relação à análise de discrepância, planejamos acrescentar outra medida à análise, a distância para o ponto médio, que acreditamos ser promissora para o trabalho.

Por fim, esclarecemos que o \href{https://chat.openai.com/auth/login?next=%2F}{ChatGPT} foi utilizado durante o trabalho para revisar o texto. O comando "Revise" foi utilizado em textos previamente escritos e depois revisado pelos autores, para garantir a concisão dos dados apresentados. 
 
