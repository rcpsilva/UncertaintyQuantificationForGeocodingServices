%--------------------------------------------------------------------------
%--------------------- Resumo em Português --------------------------------
%--------------------------------------------------------------------------

\setlength{\absparsep}{18pt} % ajusta o espaçamento dos parágrafos do resumo
\begin{resumo}

  As APIs de geocodificação online desempenham um papel significativo em aplicações que requerem informações de localização. Para garantir a qualidade dessas aplicações, é essencial avaliar a precisão das APIs utilizadas. Este estudo tem como objetivo principal avaliar a qualidade de quatro APIs de geocodificação: Google Maps, Mapbox, TomTom e Open Route Service (ORS). Para isso, foram utilizadas duas bases de dados: a primeira de Belo Horizonte e a segunda da região metropolitana de São Paulo. Além de verificar a qualidade das APIs, o trabalho também tem como objetivo encontrar uma medida de discrepância comparável com o erro e avaliar a melhor formatação de entrada para as APIs.

  Utilizamos várias métricas para a análise de qualidade, incluindo média, desvio padrão, mediana, média aparada em 5\%, taxa de resposta (proporção entre solicitações de geocodificação e respostas) e taxa de acerto (quantidade de endereços com erro menor que 150 metros). Além disso, conduzimos uma análise espacial do erro e investigamos a relação entre discrepância e erro, usando a medida de distância ao ponto médio.
  
  A maioria das APIs apresentou uma taxa de resposta alta em ambas as bases. Em relação à taxa de acerto, a maioria das APIs teve resultados de médios a altos, com exceção da API ORS, que teve resultados insatisfatórios nas duas bases de dados. No que diz respeito à relação entre discrepância e erro, identificamos uma correlação forte a muito forte com todas as APIs, exceto Google Maps, que teve uma correlação muito fraca para ambas as bases, e ORS, que teve correlação fraca para a base de São Paulo.
  
  Por fim, em relação à formatação dos dados de entrada, tivemos resultados que indicam que a melhor formatação de entrada é a que segue o código postal brasileiro e que a adição de bairro pode causar melhora na geocodificação de algumas das APIs avaliadas.
  

 \vspace{\onelineskip}
 \noindent
 \textbf{Palavras-chave}: GeoAPIs. Qualidade. 

\end{resumo}

%--------------------------------------------------------------------------
%--------------------- Resumo em Inglês --------------------------------
%--------------------------------------------------------------------------
\begin{resumo}[Abstract]
 \begin{otherlanguage*}{english}
  Online geocoding APIs play a significant role in applications that require location information. To ensure the quality of these applications, it is essential to evaluate the accuracy of the APIs used. This study aims to evaluate the quality of four geocoding APIs: Google Maps, Mapbox, TomTom, and Open Route Service (ORS). For this, two databases were used: the first from Belo Horizonte and the second from the metropolitan region of São Paulo. In addition to assessing the quality of the APIs, the study also aims to find a discrepancy measure comparable to the error and to evaluate the best input formatting for the APIs.

  We used several metrics for quality analysis, including mean, standard deviation, median, 5\% trimmed mean, response rate (proportion between geocoding requests and responses), and precision rate (number of addresses with an error less than 150 meters). Additionally, we conducted a spatial error analysis and investigated the relationship between discrepancy and error, using the distance to the midpoint measure.
  
  Most APIs showed a high response rate in both databases. Regarding the precision rate, most APIs had medium to high results, except for the ORS API, which had unsatisfactory results in both databases. Concerning the relationship between discrepancy and error, we identified a strong to very strong correlation with all APIs, except for Google Maps, which had a very weak correlation for both databases, and ORS, which had a weak correlation for the São Paulo database.
  
  Finally, concerning the input data formatting, we had results indicating that the best input formatting follows the Brazilian postal code, and adding the neighborhood can improve geocoding for some of the evaluated APIs.

   \vspace{\onelineskip}
   \noindent 
   \textbf{Keywords}: GeoAPIs. Quality
 \end{otherlanguage*}
\end{resumo}