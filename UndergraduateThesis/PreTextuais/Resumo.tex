%--------------------------------------------------------------------------
%--------------------- Resumo em Português --------------------------------
%--------------------------------------------------------------------------

\setlength{\absparsep}{18pt} % ajusta o espaçamento dos parágrafos do resumo
\begin{resumo}
Síntese do trabalho contendo um único parágrafo. O resumo deve ser feito de forma clara, concisa e seletiva de todo o texto, ressaltando o objetivo, o método, os resultados e a conclusão  \cite{NBR6028:2003}. A norma da ABNT ainda recomenda que  a primeira frase seja uma explicação do tema principal, seguindo da informação da natureza do trabalho (pesquisa experimental, pesquisa biliográfica, estudo de caso, pesquisa de campo, etc.). Apresente os objetivos (geral e específicos); justificativa e a metodologia desenvolvida. Também deve ser inserido as conclusões finais, apresentando uma síntese dos principais resultados alcançados e o valor da pesquisa no contexto acadêmico. Sugere-se entre 150 a 500 palavras.

 \vspace{\onelineskip}
 \noindent
 \textbf{Palavras-chave}: Palavra-chave 1. Palavra-chave 2. Palavra-chave 3. 

As palavras-chave devem estar separadas por ponto e finalizadas também por ponto. Devem ser escolhidos termos  que descrevem o conteúdo do trabalho.
\end{resumo}

%--------------------------------------------------------------------------
%--------------------- Resumo em Inglês --------------------------------
%--------------------------------------------------------------------------
\begin{resumo}[Abstract]
 \begin{otherlanguage*}{english}
   This is the english abstract.


   \vspace{\onelineskip}
   \noindent 
   \textbf{Keywords}: Keywords1, Keywords2, Keywords3.
 \end{otherlanguage*}
\end{resumo}