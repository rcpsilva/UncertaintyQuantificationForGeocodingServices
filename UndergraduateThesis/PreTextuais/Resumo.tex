%--------------------------------------------------------------------------
%--------------------- Resumo em Português --------------------------------
%--------------------------------------------------------------------------

\setlength{\absparsep}{18pt} % ajusta o espaçamento dos parágrafos do resumo
\begin{resumo}

  As APIs de geocodificação online desempenham um papel significativo em aplicações que requerem informações de localização. Para garantir a qualidade dessas aplicações, é essencial avaliar a precisão das APIs utilizadas. Este estudo tem como objetivo principal avaliar a qualidade de quatro APIs de geocodificação: Google Maps, Mapbox, TomTom e Open Route Service (ORS). Para isso, foram utilizadas duas bases de dados: a primeira de Belo Horizonte e a segunda da região metropolitana de São Paulo. Além de verificar a qualidade das APIs, o trabalho também tem como objetivo encontrar uma medida de discrepância comparável com o erro e avaliar a melhor formatação de entrada para as APIs.

  Utilizamos várias métricas para a análise de qualidade, incluindo média, desvio padrão, mediana, média aparada em 5\%, taxa de resposta (proporção entre solicitações de geocodificação e respostas) e taxa de acerto (quantidade de endereços com erro menor que 150 metros). Além disso, conduzimos uma análise espacial do erro e investigamos a relação entre discrepância e erro, usando a medida de distância ao ponto médio.
  
  A maioria das APIs apresentou uma taxa de resposta alta em ambas as bases. Em relação à taxa de acerto, a maioria das APIs teve resultados de médios a altos, com exceção da API ORS, que teve resultados insatisfatórios nas duas bases de dados. No que diz respeito à relação entre discrepância e erro, identificamos uma correlação forte a muito forte com todas as APIs, exceto Google Maps, que teve uma correlação muito fraca para ambas as bases, e ORS, que teve correlação fraca para a base de São Paulo.
  
  Por fim, em relação à formatação dos dados de entrada, tivemos resultados que indicam que a melhor formatação de entrada é a que segue o código postal brasileiro e que a adição de bairro pode causar melhora na geocodificação de algumas das APIs avaliadas.
  

 \vspace{\onelineskip}
 \noindent
 \textbf{Palavras-chave}: GeoAPIs. Qualidade. 

\end{resumo}

%--------------------------------------------------------------------------
%--------------------- Resumo em Inglês --------------------------------
%--------------------------------------------------------------------------
\begin{resumo}[Abstract]
 \begin{otherlanguage*}{english}
   Online geocoding APIs play a significant role in applications that require location information. To ensure the quality of these applications, it is essential to assess the accuracy of the APIs used. This study aims to evaluate the quality of five geocoding APIs implemented in TerraLAB: Google Maps, Mapbox, TomTom, Here, and Open Route Service (ORS). The evaluation was conducted based on geocoding error compared to a reference database in the metropolitan region of São Paulo.

    We used various metrics for comparative analysis, including mean, standard deviation, median, trimmed mean at 5\%, response rate (the ratio of geocoding requests to responses), and accuracy rate (the number of addresses with errors less than 150 meters). Additionally, we conducted a spatial analysis of the error and investigated the relationship between discrepancy and error using the covariance measure. Due to issues with the application collecting geocodings, this project phase focused only on the Mapbox, TomTom, and Here APIs, resulting in overall unsatisfactory performance.
    
    Most APIs exhibited a low response rate, with the highest among them falling below 90\%, which impacted the experiment's integrity. Regarding the accuracy rate, all APIs obtained values considered unsatisfactory by our research team. Furthermore, we observed significant errors that hindered spatial analysis. Concerning the relationship between discrepancy and error, we could not identify a strong correlation, possibly due to the limited number of geocodings performed.
    
    For the next phase of the project, we plan to repeat the analysis with the remaining APIs for São Paulo data and extend the evaluation to Belo Horizonte data.


   \vspace{\onelineskip}
   \noindent 
   \textbf{Keywords}: GeoAPIs. Quality
 \end{otherlanguage*}
\end{resumo}