%--------------------------------------------------------------------------
%--------------------- Resumo em Português --------------------------------
%--------------------------------------------------------------------------

\setlength{\absparsep}{18pt} % ajusta o espaçamento dos parágrafos do resumo
\begin{resumo}
  As APIs de geocodificação online desempenham um papel significativo em aplicações que requerem informações de localização. Para garantir a qualidade dessas aplicações, é essencial avaliar a precisão das APIs utilizadas. Este estudo tem como objetivo avaliar a qualidade de cinco APIs de geocodificação implementadas no TerraLAB: Google Maps, Mapbox, TomTom, Here e Open Route Service (ORS). A avaliação foi realizada com base no erro de geocodificação em comparação com uma base de dados de referência na região metropolitana de São Paulo.
  Utilizamos várias métricas para a análise comparativa, incluindo média, desvio padrão, mediana, média aparada em 5\%, taxa de resposta (proporção entre solicitações de geocodificação e respostas) e taxa de acerto (quantidade de endereços com erro menor que 150 metros). Além disso, conduzimos uma análise espacial do erro e investigamos a relação entre discrepância e erro, usando a medida de covariância. Devido a problemas na aplicação que coleta as geocodificações, esta etapa do projeto se concentrou apenas nas APIs Mapbox, TomTom e Here, resultando em um desempenho geral insatisfatório.
  A maioria das APIs apresentou uma taxa de resposta baixa, com a maior delas ficando abaixo de 90\%, o que impactou a integridade do experimento. Em relação à taxa de acerto, todas as APIs obtiveram valores considerados insatisfatórios pela nossa equipe de pesquisa. Além disso, observamos a ocorrência de erros significativos que prejudicaram a análise espacial. No que diz respeito à relação entre discrepância e erro, não pudemos identificar uma correlação forte, possivelmente devido ao número limitado de geocodificações realizadas.
  Para a próxima fase do projeto, planejamos repetir a análise com as APIs restantes para os dados de São Paulo e estender a avaliação para os dados de Belo Horizonte
  
 \vspace{\onelineskip}
 \noindent
 \textbf{Palavras-chave}: GeoAPIs. Qualidade. 

\end{resumo}

%--------------------------------------------------------------------------
%--------------------- Resumo em Inglês --------------------------------
%--------------------------------------------------------------------------
\begin{resumo}[Abstract]
 \begin{otherlanguage*}{english}
   This is the english abstract.


   \vspace{\onelineskip}
   \noindent 
   \textbf{Keywords}: Keywords1, Keywords2, Keywords3.
 \end{otherlanguage*}
\end{resumo}