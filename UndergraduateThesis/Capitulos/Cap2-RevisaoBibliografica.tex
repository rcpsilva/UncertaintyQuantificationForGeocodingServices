\chapter{Avaliação da Geocodificação} \label{RevisaoBibliografica}

%Este capítulo deve apresentar uma contextualização da sua pesquisa com um resumo das discussões já feitas por outros autores sobre o assunto abordado e os conceitos principais relativos ao tema. O nome deste capítulo  de \textbf{Revisão Bibliográfica} ou \textbf{Embasamento Teórico} deve ser acordado com seu orientador.

%A revisão bibliográfica é a base que sustenta qualquer pesquisa científica e  é indispensável para a delimitação do problema em um projeto de pesquisa,  para obter uma ideia precisa sobre o estado atual dos conhecimentos sobre um tema, sobre suas lacunas e sobre a contribuição da investigação para o desenvolvimento do conhecimento \cite{marconi2003}. 

%Para a escrita deste capítulo, as citações e referências devem estar de acordo com a norma \cite{NBR6023:2002}, que destina-se a orientar a preparação e compilação das  referências bibliográficas de todo o documento.

%\section{Trabalhos Relacionados}

%Descreva os principais trabalhos realizados por outros autores sobre a temática escolhida para ser desenvolvida, apresentando os conceitos mais importantes, justificativas e características sobre o tema, do ponto de vista da análise feita pelos autores. 

%É importante destacar, no contexto da pesquisa, quais os resultados já alcançados e os respectivos responsáveis e se possível uma análise  dos trabalhos consultados. Finalize a seção comparando a sua proposta de pesquisa com os trabalhos citados, destacando as semelhanças (caso existam) e a sua contribuição (o que pretende desenvolver).


\section{Geocodificação}

%\section{Qualidade de dados}

\section{APIs de Geocodificação}







