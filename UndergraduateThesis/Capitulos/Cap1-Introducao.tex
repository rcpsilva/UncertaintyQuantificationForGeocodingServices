\chapter{Introdução} \label{Introducao}

A representação computacional de endereços físicos é uma necessidade constante. Para realizar essa tarefa, é preciso traduzir os endereços, geralmente expressos em sequências de palavras, para coordenadas em um plano. A forma mais comum de representar essas coordenadas é através de latitude e longitude. Esse processo de tradução é conhecido como Geocodificação e, como qualquer processo, está sujeito a erros.

O presente trabalho tem como objetivo avaliar a qualidade da geocodificação de diferentes tecnologias em duas grandes cidades do Brasil. Para isso, será utilizada uma base de referência para analisar o erro, a discrepância e a acurácia de cada uma das tecnologias empregadas.

\section{Justificativa}

A representação de endereços é amplamente utilizada em diversas aplicações, desde funções mais óbvias, como softwares de roteamento e pesquisa de endereços, até aplicações mais complexas, como o uso de informações geográficas em sistemas de inteligência artificial para indicação de produtos.

Nesse contexto, é de extrema importância que os softwares responsáveis pela tradução de endereços apresentem alta qualidade em suas respostas. Compreender a qualidade das APIs utilizadas é fundamental para tomar decisões adequadas de acordo com a aplicação em questão. Além disso, conhecer as falhas dessas APIs é relevante para aprimorar as aplicações que dependem de geocodificação.

\section{Objetivos}

O principal objetivo deste trabalho é avaliar o erro, a discrepância e a acurácia de cinco APIs utilizadas no laboratório de pesquisa e capacitação em desenvolvimento de software - \href{http://www2.decom.ufop.br/terralab/}{TerraLAB}. As APIs em análise são: Google Maps, TomTom, Open Route Service (ORS), Mapbox e Here. O erro será analisado quanto às respostas fornecidas pelas APIs diferirem do esperado. A discrepância medirá o nível de discordância entre as APIs. Por fim, a acurácia será utilizada para verificar a precisão das respostas fornecidas pelas APIs.

Uma parte essencial do trabalho é compreender os pontos onde essas APIs apresentam falhas, e, portanto, a análise espacial dessas medidas terá grande destaque na pesquisa.

\section{Organização do Trabalho}

Um parágrafo fazendo uma descrição dos capítulos restantes do documento. 

\subsection{Estrutura da Monografia}

Segue uma \textbf{sugestão} para a estrutura da monografia: 

\begin{description}
   \item[Capítulo 1:] Introdução.
   \item[Capítulo \ref{RevisaoBibliografica}:] Revisão Bibliográfica/ Embasamento Teórico (com o referencial teórico e trabalhos relacionados).
   \item[Capítulo \ref{desenvolvimento}:] Metodologia ou Desenvolvimento (material e métodos).
   \item[Capítulo \ref{resultado}:] Resultados e Discussões.
   \item[Capítulo \ref{conclusao}:] Conclusão (e trabalhos futuros).
\end{description}


 









