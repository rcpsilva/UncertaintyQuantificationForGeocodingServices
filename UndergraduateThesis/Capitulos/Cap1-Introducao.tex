\chapter{Introdução} \label{Introducao}


Este capítulo é a parte inicial do trabalho. Deve ser apresentado, com apoio da literatura, uma síntese do tema central do trabalho indicando o problema da pesquisa; os objetivos, a justificativa e a organização do trabalho. 


É importante ressaltar que um \textbf{trabalho de conclusão de curso} (TCC) da graduação é um ``documento que apresenta o resultado de estudo, devendo expressar conhecimento do assunto escolhido, que deve ser obrigatoriamente emanado da disciplina, módulo, estudo independente, curso, programa, e outros ministrados'' \cite{NBR14724:2011}  e  deve ser feito sob a coordenação de um orientador e/ou coorientador.

\section{Justificativa}

Justificativa para a realização do trabalho, situando sua relevância no contexto da sua área de formação e sua importância para o avanço do conhecimento. A justifica também precisa destacar a motivação pessoal para a escolha do tema estudado.

\section{Objetivos}

Descrever o objetivo principal e os objetivos específicos da pesquisa. Os objetivos constituem a finalidade de um trabalho científico. O objetivo principal é mais amplo e deve descrever de forma clara e sucinta qual a meta que se deseja atingir (uma proposta que solucione um problema; uma proposta de melhoria; uma análise de uma situação, entre outros exemplos). 

Para cumprir o objetivo principal muitas vezes é necessário delimitá-lo em partes menores, ou seja, em objetivos específicos. Os objetivos específicos devem ser partes do trabalho, que após cumpridos pelo pesquisador, atinge-se  o objetivo principal do trabalho. Devem vir listados, apresentados por tópicos e separados por ponto-e-vírgula.




\section{Organização do Trabalho}

Um parágrafo fazendo uma descrição dos capítulos restantes do documento. 

\subsection{Estrutura da Monografia}

Segue uma \textbf{sugestão} para a estrutura da monografia: 

\begin{description}
   \item[Capítulo 1:] Introdução.
   \item[Capítulo \ref{RevisaoBibliografica}:] Revisão Bibliográfica/ Embasamento Teórico (com o referencial teórico e trabalhos relacionados).
   \item[Capítulo \ref{desenvolvimento}:] Metodologia ou Desenvolvimento (material e métodos).
   \item[Capítulo \ref{resultado}:] Resultados e Discussões.
   \item[Capítulo \ref{conclusao}:] Conclusão (e trabalhos futuros).
\end{description}


 









