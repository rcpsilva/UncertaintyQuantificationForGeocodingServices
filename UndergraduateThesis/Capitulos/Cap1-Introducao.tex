\chapter{Introdução} \label{Introducao}

Constantemente é necessário que representemos endereços físicos de forma computacional. Para que isso seja feito, é necessário traduzir o endereço, que geralmente é representado por uma sequência de palavras, para coordenadas de um plano. A forma mais comum dessas coordenadas é a representação por latitude e longitude. 
Ess processo de tradução do endereço se chama Geocodificação. E como todos os processos, ele possui erro. 
O presente trabalho pretende avaliar a qualidade da geocodificação de diferentes tecnologias para duas cidades grandes do Brasil, usando um base de referência e analisando o erro, a discrepância e a acurácia de cada uma das tecnologias.

\section{Justificativa}

A representação dos endereços é utilizada em diversas aplicações, desde as mais obvias, como softwares de roteamento e pesquisa de enderços. Até aplicacações mais complexas como a indicação de produtos por meio de inteligência artificial, que por vezes utiliza de informações geográficas. 
Sendo assim, é de extrema importância que os softwares que realizam a tarefa de traduzir endereços façam isso com a melhor qualidade possível. É importante também 

\section{Objetivos}

Descrever o objetivo principal e os objetivos específicos da pesquisa. Os objetivos constituem a finalidade de um trabalho científico. O objetivo principal é mais amplo e deve descrever de forma clara e sucinta qual a meta que se deseja atingir (uma proposta que solucione um problema; uma proposta de melhoria; uma análise de uma situação, entre outros exemplos). 

Para cumprir o objetivo principal muitas vezes é necessário delimitá-lo em partes menores, ou seja, em objetivos específicos. Os objetivos específicos devem ser partes do trabalho, que após cumpridos pelo pesquisador, atinge-se  o objetivo principal do trabalho. Devem vir listados, apresentados por tópicos e separados por ponto-e-vírgula.

 


\section{Organização do Trabalho}

Um parágrafo fazendo uma descrição dos capítulos restantes do documento. 

\subsection{Estrutura da Monografia}

Segue uma \textbf{sugestão} para a estrutura da monografia: 

\begin{description}
   \item[Capítulo 1:] Introdução.
   \item[Capítulo \ref{RevisaoBibliografica}:] Revisão Bibliográfica/ Embasamento Teórico (com o referencial teórico e trabalhos relacionados).
   \item[Capítulo \ref{desenvolvimento}:] Metodologia ou Desenvolvimento (material e métodos).
   \item[Capítulo \ref{resultado}:] Resultados e Discussões.
   \item[Capítulo \ref{conclusao}:] Conclusão (e trabalhos futuros).
\end{description}


 









