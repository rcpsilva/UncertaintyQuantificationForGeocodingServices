\chapter{Introdução} \label{Introducao}


% Contextualizar o problema (De onde vem o problema)?

% Definir o problema com precisão
   % Qual API erra mais?
   % Existe algum padrão espacial no erro?
   % Variância entre as APIs representa o erro?

% Motivação (Por quê estes problemas/perguntas são relevantes?)

% Revisão da literatura (O que já foi feito sobre o problema e o que falta fazer?)

% Objetivo (O que você pretente atingir?)
% Objetivos específicos (objetivos intermediários) 


% OBS: Toda afirmação deve ter uma referência.
Segundo o \cite{Zamberg2009} endereço é a principal forma de de conceitualizar localização no mundo atual. Isso se deve ao fato dos endereços serem utilizados em diversas aplicações de diferentes campos de estudo, como na saúde \cite{AmericaJournal2001,Kypri2009,Mazumdar2008}, nas ciências sociais \cite{Chow2011}, na análise de criminal ou judiciária \cite{Olligschlaeger1998}, na análise ambiental \cite{Gilboa2006}, na ciência da computação \cite{Zamberg2009}, na economia \cite{Whitsel2006} entre outras.

Para isso, é necessário gerar a representação computacional do endereço, de forma com que as aplicações possam utilizá-lo. A representaçãomais comum segundo \cite{Zamberg2009} é a representação por meio de coordenadas x e y em um plano, geralmente a medida é latitude e longitude. O processo de tranformação em um endereço nessas coordenadas é chamado de Geocodificação. Para \cite{Zamberg2009} esse processo consiste em 3 etapas:
\begin{itemize}
   \item Processamento do endereço de entrada: o endereço será lido, dividido em componentes (rua, número, bairro, etc), padronizado, cada campo é atribuído a uma categoria e por fim, serão indexadas as categorias necessárias; 
   \item Busca na base referência: de acordo com o algoritmo escolhido, será realizada uma busca na base referência afim de selecionar e classificar potenciais canditados para resposta;
   \item Seleção do(s) canditado(s) para resposta: com a busca realizada será feita uma análise da classificação gerada por ela e serão escolhidos os melhores canditados.
\end{itemize}
Todas as etapas podem imprimir erro no resultado final, fazendo com que a geocodificação não seja confiável ou tenha uma qualidade ruim. Fazendo com que muitos usários tenham que realizar o trabalho de tradução manualmente, geralmente por meio de GPS e mapeamento em campo, o que é um trabalho custoso e pouco produtivo. No artigo de \cite{Zamberg2009} 

O presente trabalho tem como objetivo avaliar a qualidade da geocodificação de diferentes tecnologias em duas grandes cidades do Brasil. Para isso, será utilizada uma base de referência para analisar o erro, a discrepância e a acurácia de cada uma das tecnologias empregadas.

\section{Justificativa}

A representação de endereços é amplamente utilizada em diversas aplicações, desde funções mais óbvias, como softwares de roteamento e pesquisa de endereços, até aplicações mais complexas, como o uso de informações geográficas em sistemas de inteligência artificial para indicação de produtos.

Nesse contexto, é de extrema importância que os softwares responsáveis pela tradução de endereços apresentem alta qualidade em suas respostas. Compreender a qualidade das APIs utilizadas é fundamental para tomar decisões adequadas de acordo com a aplicação em questão. Além disso, conhecer as falhas dessas APIs é relevante para aprimorar as aplicações que dependem de geocodificação.

\section{Objetivos}

O principal objetivo deste trabalho é avaliar o erro, a discrepância e a acurácia de cinco APIs utilizadas no laboratório de pesquisa e capacitação em desenvolvimento de software - TerraLAB. As APIs em análise são: Google Maps, TomTom, Open Route Service (ORS), Mapbox e Here. O erro será analisado quanto às respostas fornecidas pelas APIs diferirem do esperado. A discrepância medirá o nível de discordância entre as APIs. Por fim, a acurácia será utilizada para verificar a precisão das respostas fornecidas pelas APIs.

Uma parte essencial do trabalho é compreender os pontos onde essas APIs apresentam falhas, e, portanto, a análise espacial dessas medidas terá grande destaque na pesquisa.

\section{Organização do Trabalho}

Um parágrafo fazendo uma descrição dos capítulos restantes do documento. 

\subsection{Estrutura da Monografia}

Segue uma \textbf{sugestão} para a estrutura da monografia: 

\begin{description}
   \item[Capítulo 1:] Introdução.
   \item[Capítulo \ref{RevisaoBibliografica}:] Revisão Bibliográfica/ Embasamento Teórico (com o referencial teórico e trabalhos relacionados).
   \item[Capítulo \ref{desenvolvimento}:] Metodologia ou Desenvolvimento (material e métodos).
   \item[Capítulo \ref{resultado}:] Resultados e Discussões.
   \item[Capítulo \ref{conclusao}:] Conclusão (e trabalhos futuros).
\end{description}


 









