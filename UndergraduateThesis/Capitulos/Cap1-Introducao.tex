\chapter{Introdução} \label{Introducao}

Constantemente é necessário que representemos endereços físicos de forma computacional. Para que isso seja feito, é necessário traduzir o endereço, que geralmente é representado por uma sequência de palavras, para coordenadas de um plano. A forma mais comum dessas coordenadas é a representação por latitude e longitude. 
Ess processo de tradução do endereço se chama Geocodificação. E como todos os processos, ele possui erro. 
O presente trabalho pretende avaliar a qualidade da geocodificação de diferentes tecnologias para duas cidades grandes do Brasil, usando um base de referência e analisando o erro, a discrepância e a acurácia de cada uma das tecnologias.

\section{Justificativa}

A representação dos endereços é utilizada em diversas aplicações, desde as mais obvias, como softwares de roteamento e pesquisa de enderços. Até aplicacações mais complexas como a indicação de produtos por meio de inteligência artificial, que por vezes utiliza de informações geográficas. 
Sendo assim, é de extrema importância que os softwares que realizam a tarefa de traduzir endereços façam isso com a melhor qualidade possível. É importante também ter consciência da qualidade das APIs para conseguir tomar a melhor decisão de acordo com a aplicação. 
Outro ponto relevante é a melhoria das aplicações em que se utiliza os endereços geocodificados, temos que saber onde e quando as APIs falham para conseguirmos fazer aplicações que contornem isso. 

\section{Objetivos}

O principal objetivo do trabalho é avaliar o erro, a discrepância e a acurácia das cinco APIs utilizadas no laboratório de pesquisa e capacitação em desenvolvimento de software - \href{https://www.exemplo.com}{TerraLAB}. As APIs utilizadas são: Google Maps, TomTom, Open Route Service (ORS), Mapbox e Here. O erro é a análise de quanto as APIs dão a resposta fora do que é esperado. A discrepânciadiz o quanto a APIs discordam entre si. Por fim, a acurácia mede o quanto as APIs realmente acertam quando dizem que acertaram. 

Um ponto importanteno trabalho é entender onde essas APIs erram, então o trabalho terá bastente foco na análise espacial dessas medidas. 


\section{Organização do Trabalho}

Um parágrafo fazendo uma descrição dos capítulos restantes do documento. 

\subsection{Estrutura da Monografia}

Segue uma \textbf{sugestão} para a estrutura da monografia: 

\begin{description}
   \item[Capítulo 1:] Introdução.
   \item[Capítulo \ref{RevisaoBibliografica}:] Revisão Bibliográfica/ Embasamento Teórico (com o referencial teórico e trabalhos relacionados).
   \item[Capítulo \ref{desenvolvimento}:] Metodologia ou Desenvolvimento (material e métodos).
   \item[Capítulo \ref{resultado}:] Resultados e Discussões.
   \item[Capítulo \ref{conclusao}:] Conclusão (e trabalhos futuros).
\end{description}


 









