\chapter{Considerações Finais} \label{consideracoes}

O presente trabalho apresentou uma análise da qualidade das APIs Mapbox, TomTom e Here para os dados disponibilizados pelo CEM - Centro de estudos da Metrópole \cite{cem}. Devido a problemas no Crawler, que é a aplicação que solicita e coleta a geocodificação, tivemos poucas respostas e estas foram insatisfatórias. A conclusão atual é de que as API cometem muitos erros graves e que não há relação clara entre a discrepância e o erro. 
Porém, quaisquer conclusões tiradas a partir desse estudo são inviesadas a partir do momento em que não temos dados o suficiente e estes são dados específicos. Há de se lembrar que a base de dados possui apenas endereços de escola, não tenho uma diversadade de imóveis, localizados na região metropolitana de São de Paulo, não tendo uma diversidade de localidades. 
Sendo assim, é necessária a repetição do experimento com um maior montante de dados. 

Para a próxima etapa do trabalho, iremos repetir os experimentos apresentados com uma nova solicitação de geocodificação além de incluir as APIs faltantes, Google Maps e Open Route Service. Acreditamos que repetindo o experimento possamos ver realmente o comportamento do erro e poderemos comparar os resultados com APIs já consolidadas na academia, a exemplo do Google Maps. Além disso, iremos fazer toda a análise para uma amostra significativa da base de dados da \cite{prodabel}. A amostra conta com 85mil endereços distribuídos no espaço. A expectativa é que com a maior quantidade de endereços poderemos vizualizar o comportamento mais claramente. 
Em relação a análise de discrepância iremos acrescentar outra medida na análise, a distância para o ponto médio. Acreditamos que essa métrica é promissora para o trabalho. 

Por fim, esclarecemos que o \href{https://chat.openai.com/auth/login?next=%2F}{ChatGPT} foi utilizado durante o trabalho para revisar o texto. O comando "Revise" foi utilizado em textos previamente escritos e depois revisado pelos autores, para garantir a concisão dos dados apresentados. 
 