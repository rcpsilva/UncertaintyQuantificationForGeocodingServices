\chapter{Considerações Finais} \label{consideracoes}

O presente trabalho apresentou uma avaliação das APIs Mapbox, TomTom, Open Route Service e Google Maps para as cidades de Belo Horizonte e São Paulo.

O trabalho tinha três objetivos principais. O primeiro foi realizar uma análise da qualidade da geocodificação das APIs, utilizando o erro como medida para aferir essa qualidade. O segundo foi verificar se uma medida de discrepância se relaciona com o erro da geocodificação, de modo que este possa ser substituído. E o terceiro foi verificar se a formatação da entrada para a geocodificação pode impactar na qualidade da mesma.

Em relação ao primeiro objetivo, as APIs que apresentaram melhores resultados foram Mapbox, Google Maps e TomTom. O desempenho dessas três APIs variou entre as bases de dados, sendo a ordem, do melhor desempenho para o pior, Mapbox, Google Maps e TomTom para os dados de Belo Horizonte; e Google Maps, TomTom e Mapbox para os dados de São Paulo. Por outro lado, o desempenho da API ORS foi constante, apresentando um desempenho insatisfatório em ambas as bases.

Considerando os resultados de cada base individualmente, a base de Belo Horizonte obteve resultados melhores em geral, com taxas de acerto altas para a maioria das APIs. Já a base de São Paulo teve resultados de médios a baixos para as APIs. Isso pode ter ocorrido devido a características particulares das bases ou problemas na aplicação que solicita e coleta as geocodificações. Mais análises são necessárias para chegar a uma conclusão concreta em relação a isso.

Para o segundo objetivo, foi escolhida a distância ao ponto médio como medida de discrepância. Ela se mostrou promissora como uma medida substituta ao erro para todas as APIs, com exceção da Google Maps, para ambas as bases, e da ORS, para a base de São Paulo. Para aquelas APIs onde há indicativo de relação, foram obtidas correlações fortes a muito fortes entre a medida e o erro. Esse resultado é um forte indicativo de que a medida pode ser considerada também uma medida de qualidade da geocodificação para essas APIs. Porém, mais estudos são necessários para confirmar essa hipótese e comparar os resultados com outras medidas para verificar se existem outras com desempenho semelhante ou superior. Em relação àquelas em que não houve relação, deve-se investigar se isso ocorre devido aos algoritmos empregados por elas ou se existe outra medida que apresenta essa relação.

Por fim, para alcançar o terceiro objetivo, foi realizada uma análise comparativa da qualidade da geocodificação de 15 experimentos, de forma similar à feita para o primeiro objetivo. Esses experimentos se diferenciavam entre si pela ordem em que as informações do endereço eram posicionadas na entrada da API. Para a base de São Paulo, foram realizados 5 experimentos, e para a base de Belo Horizonte foram realizados 10, que são os mesmos 5 de São Paulo, com a adição de mais 5 experimentos incluindo a informação de bairro. Para ambas as bases e a maioria das APIs, os experimentos com o formato de entrada similar ao padrão do código postal brasileiro mostraram melhores resultados. Isso nos leva a crer que esse formato é o mais indicado para a entrada das APIs de geocodificação. Em relação à adição de bairro, os resultados foram conflitantes. Para algumas APIs, houve melhora ao adicionar o bairro; para outras, não houve diferença; e, para outras, houve uma piora. Em geral, a adição do bairro parece ser uma boa opção para a geocodificação, porém mais análises são necessárias.

Como trabalhos futuros, pode-se expandir as análises de qualidade, trabalhando com mais bases de dados de outras cidades importantes brasileiras ou com bases de locais diversos. Além disso, poderiam ser expandidas as análises de discrepância, adicionando outras medidas e comparando com os resultados obtidos com a distância ao ponto médio. Em relação aos experimentos, pode-se avaliar outras bases com a informação de bairro, para verificar com mais profundidade o impacto dessa adição, ou adicionar outras informações não utilizadas neste trabalho, como CEP.

Por fim, esclarecemos que o \href{https://chat.openai.com/auth/login?next=%2F}{ChatGPT} foi utilizado durante o trabalho para revisar o texto. O comando "Revise" foi utilizado em textos previamente escritos e depois revisado pelos autores, para garantir a concisão dos dados apresentados. 
 
