\chapter{Considerações Finais} \label{consideracoes}

Neste capítulo deve ser explicitado se todos os objetivos descritos na introdução foram atingidos e ressaltar a contribuição do trabalho para o meio acadêmico.

São apresentados de forma sucinta os resultados obtidos e um fechamento de todo trabalho desenvolvido.



%%====== Section ========%
\section{Conclusão}\label{conclusao}

Em resumo, nesta seção devem ser apresentadas as considerações finais do trabalho. Faça uma recapitulação a respeito de cada um dos objetivos específicos, sintetize os resultados obtidos e conclua se o objetivo principal do trabalho foi alcançado.


%%====== Section ========%
\section{Trabalhos Futuros}\label{trabalhosFuturos}

Apresente propostas de continuidade do seu trabalho.

%%====== Section ========%
\section{Publicações Realizadas}\label{publicacoes}

Caso o trabalho tenha originado publicações é válido acrescentar essa informação, visto que pode creditar ainda mais o estudo. Assim, elas devem ser apresentadas na forma de uma subseção do capítulo conclusão. Por exemplo: 

Os trabalhos seguintes, que foram originados das metodologias propostas, foram aceitos para apresentação em conferências nacionais:

\begin{enumerate}
   \item Autor. Título do Artigo. Cidade: Conferência, Ano.
 \end{enumerate}

